\documentclass[a4paper, 12pt]{article}
\usepackage[utf8]{inputenc}
\usepackage[T2A]{fontenc}
\usepackage[russian, english]{babel}
\usepackage{geometry}

\geometry{left=0.7cm,right=0.7cm,top=0.7cm,bottom=0.7cm}

\begin{document}

\begin{center}
	\textbf{\large 1. Титульный слайд}
\end{center}	



\begin{center}
	\textbf{\large 2. Состояние проблемы и основные трудности в предметной области}
\end{center}	



\begin{center}
	\textbf{\large 3. Цель исследования и практическая значимость}
\end{center}	

Чем решили не заниматься: не выбирать состав команд, не заниматься интерпретацией команд, не очень использовали возможности допобучения хотя немного коснулись этого.

Требования к бортовому оборудованию являются очень высокими – для систем, не влияющих на безопасность полёта, равно $10^{-3}$ ошибок в час, а для влияющих ещё больше.

Для любых система распознавания обойти человека – это большое достижение.




\begin{center}
	\textbf{\large 4. Научная новизна}
\end{center}	



\begin{center}
	\textbf{\large 5. Положения, выносимые на защиту}
\end{center}	



\begin{center}
	\textbf{\large 6. Статьи и конференции}
\end{center}	



\begin{center}
	\textbf{\large 7. Тестовая база речевых данных}
\end{center}	

3 слова для более простой задачи для переключения экранов.

Общий объем записей на 1-2 порядка меньше общепринятой практики.

Заказчики систем распознавания заявили, что готовы переучивать пилотов и менять названия команд, хотя на иностранных самолётах такое применяется.

Интерпретация тут особо не важна, так как предполагалось, что каждая команда отвечает за определённое действие. 



\begin{center}
	\textbf{\large 8. Получение параметрического портрета из звукового сигнала}
\end{center}	

Высокочастотные компоненты особо значимы в процессе распознавания. После Фурье получаем результат оценок спектральных плотностей. Логарифмирование повышает чувствительность к элементам с малой амплитудой.

Возможность применения БПФ основана на гипотезе стационарности сигнала на малом временном интервале, что подтверждается обширным опытом прошлых исследований.

Используемое пространство признаков – это пространство параметрических портретов. Да, оно большое, но там различия только в части измерений и они очень скоррелированы.



\begin{center}
	\textbf{\large 9. Пример – входной сигнал и параметрический портрет 35 х 48}
\end{center}	



\begin{center}
	\textbf{\large 10. Разбиение на фонетически однородные части – функционалы}
\end{center}	

Есть нормировка на число интервалов и на число частей. Также для дисперсии вычисляется корень. Коэффициенты подобраны эмпирическим способом, чтобы все критерии давали примерно одинаковый вес.



\begin{center}
	\textbf{\large 11. Стандартная схема динамического программирования}
\end{center}	



\begin{center}
	\textbf{\large 12. Модифицированная схема динамического программирования}
\end{center}	



\begin{center}
	\textbf{\large 13. Разбиение на фонетически однородные части – примеры}
\end{center}	



\begin{center}
	\textbf{\large 14. Формирование эталона с помощью метода главных компонент – теория}
\end{center}	



\begin{center}
	\textbf{\large 15. Формирование эталона с помощью метода главных компонент – результаты 1}
\end{center}	



\begin{center}
	\textbf{\large 16. Формирование эталона с помощью метода главных компонент – результаты 2}
\end{center}	



\begin{center}
	\textbf{\large 17. Сжатие с помощью полиномов Чебышёва – теория}
\end{center}	



\begin{center}
	\textbf{\large 18. Сжатие с помощью полиномов Чебышёва – результаты}
\end{center}	



\begin{center}
	\textbf{\large 19. Распознавание несколькими эталонами – метод Байеса 1}
\end{center}	



\begin{center}
	\textbf{\large 20. Распознавание несколькими эталонами – метод Байеса 2}
\end{center}	



\begin{center}
	\textbf{\large 21. Распознавание несколькими эталонами – метод комитетов}
\end{center}	



\begin{center}
	\textbf{\large 22. Распознавание несколькими эталонами – результаты}
\end{center}	



\begin{center}
	\textbf{\large 23. Распознавание свёрточными нейронными сетями – структура сети}
\end{center}	



\begin{center}
	\textbf{\large 24. Свёрточные нейронные сети – распознавание без шума}
\end{center}	



\begin{center}
	\textbf{\large 25. Свёрточные нейронные сети – распознавание с шумом}
\end{center}	



\begin{center}
	\textbf{\large 26. Свёрточные нейронные сети – распознавание по себе}
\end{center}	



\begin{center}
	\textbf{\large 27. Результаты работы – сравнение с эталоном}
\end{center}	



\begin{center}
	\textbf{\large 28. Результаты работы – нейронные сети}
\end{center}	

Кажется, что стандартные отклонения больше характеризуют размер выборки, чем надёжность метода.



\begin{center}
	\textbf{\large 29. Доклад окончен}
\end{center}	


	
\end{document}
%%% Основные сведения %%%
\newcommand{\thesisAuthorLastName}{Полиев}
\newcommand{\thesisAuthorOtherNames}{Александр Владимирович}
\newcommand{\thesisAuthorInitials}{А.\,В.}
\newcommand{\thesisAuthor}             % Диссертация, ФИО автора
{%
    \texorpdfstring{% \texorpdfstring takes two arguments and uses the first for (La)TeX and the second for pdf
        \thesisAuthorLastName~\thesisAuthorOtherNames% так будет отображаться на титульном листе или в тексте, где будет использоваться переменная
    }{%
        \thesisAuthorLastName, \thesisAuthorOtherNames% эта запись для свойств pdf-файла. В таком виде, если pdf будет обработан программами для сбора библиографических сведений, будет правильно представлена фамилия.
    }
}
\newcommand{\thesisAuthorShort}        % Диссертация, ФИО автора инициалами
{\thesisAuthorInitials~\thesisAuthorLastName}
\newcommand{\thesisUdk}                % Диссертация, УДК
{004.934:629.7.05}
\newcommand{\thesisTitle}              % Диссертация, название
{Разработка алгоритмов для распознавания команд речевого интерфейса кабины пилота}
\newcommand{\thesisSpecialtyNumber}    % Диссертация, специальность, номер
{05.13.01}
\newcommand{\thesisSpecialtyTitle}     % Диссертация, специальность, название (название взято с сайта ВАК для примера)
{Системный анализ, управление и обработка информации}
%% \newcommand{\thesisSpecialtyTwoNumber} % Диссертация, вторая специальность, номер
%% {\todo{XX.XX.XX}}
%% \newcommand{\thesisSpecialtyTwoTitle}  % Диссертация, вторая специальность, название
%% {\todo{Теория и~методика физического воспитания, спортивной тренировки,
%% оздоровительной и~адаптивной физической культуры}}
\newcommand{\thesisDegree}             % Диссертация, ученая степень
{кандидата технических наук}
\newcommand{\thesisDegreeShort}        % Диссертация, ученая степень, краткая запись
{канд.~техн.~наук}
\newcommand{\thesisCity}               % Диссертация, город написания диссертации
{Москва}
\newcommand{\thesisYear}               % Диссертация, год написания диссертации
{2020}
\newcommand{\thesisOrganization}       % Диссертация, организация
{Федеральное государственное автономное \\
образовательное учреждение высшего образования \\
<<Московский физико-технический институт \\
(национальный исследовательский университет)>> \\
Кафедра управляющих и информационных систем}
\newcommand{\thesisOrganizationShort}  % Диссертация, краткое название организации для доклада
{МФТИ}

\newcommand{\thesisInOrganization}     % Диссертация, организация в предложном падеже: Работа выполнена в ...
{кафедре управляющих и информационных систем
Московского физико-технического института
(национального исследовательского университета)}

%% \newcommand{\supervisorDead}{}           % Рисовать рамку вокруг фамилии
\newcommand{\supervisorFio}              % Научный руководитель, ФИО
{Корсун Олег Николаевич}
\newcommand{\supervisorRegalia}          % Научный руководитель, регалии
{доктор технических наук, профессор}
\newcommand{\supervisorFioShort}         % Научный руководитель, ФИО
{О.\,Н.~Корсун}
\newcommand{\supervisorRegaliaShort}     % Научный руководитель, регалии
{д-р~техн.~наук,~проф.}
\newcommand{\supervisorJobPlace}      	 % Научный руководитель, место работы
{ФГУП <<Государственный научно-исследовательский институт авиационных систем>>}
\newcommand{\supervisorJobPost}          % Научный руководитель, должность
{начальник лаборатории}

%% \newcommand{\supervisorTwoDead}{}        % Рисовать рамку вокруг фамилии
%% \newcommand{\supervisorTwoFio}           % Второй научный руководитель, ФИО
%% {\todo{Фамилия Имя Отчество}}
%% \newcommand{\supervisorTwoRegalia}       % Второй научный руководитель, регалии
%% {\todo{уч. степень, уч. звание}}
%% \newcommand{\supervisorTwoFioShort}      % Второй научный руководитель, ФИО
%% {\todo{И.\,О.~Фамилия}}
%% \newcommand{\supervisorTwoRegaliaShort}  % Второй научный руководитель, регалии
%% {\todo{уч.~ст.,~уч.~зв.}}

\newcommand{\opponentOneFio}           % Оппонент 1, ФИО
{Никульчев Евгений Витальевич}
\newcommand{\opponentOneRegalia}       % Оппонент 1, регалии
{доктор технических наук, профессор}
\newcommand{\opponentOneJobPlace}      % Оппонент 1, место работы
{ФГБОУ ВО <<МИРЭА --- Российский технологический университет>>}
\newcommand{\opponentOneJobPost}       % Оппонент 1, должность
{профессор кафедры управления и моделирования систем}

\newcommand{\opponentTwoFio}           % Оппонент 2, ФИО
{Чучупал Владимир Яковлевич}
\newcommand{\opponentTwoRegalia}       % Оппонент 2, регалии
{кандидат физико-математических наук}
\newcommand{\opponentTwoJobPlace}      % Оппонент 2, место работы
{ФГУ <<ФИЦ <<Информатика и управление>> РАН>>}
\newcommand{\opponentTwoJobPost}       % Оппонент 2, должность
{ведущий научный сотрудник}

%% \newcommand{\opponentThreeFio}         % Оппонент 3, ФИО
%% {\todo{Фамилия Имя Отчество}}
%% \newcommand{\opponentThreeRegalia}     % Оппонент 3, регалии
%% {\todo{кандидат физико-математических наук}}
%% \newcommand{\opponentThreeJobPlace}    % Оппонент 3, место работы
%% {\todo{Основное место работы c длинным длинным длинным длинным названием}}
%% \newcommand{\opponentThreeJobPost}     % Оппонент 3, должность
%% {\todo{старший научный сотрудник}}

\newcommand{\leadingOrganizationTitle} % Ведущая организация, дополнительные строки. Удалить, чтобы не отображать в автореферате
{ФГБУН Санкт-Петербургский институт информатики и автоматизации РАН}

\newcommand{\defenseDate}              % Защита, дата
{<<21>> мая 2020 года в 14:00 часов}
\newcommand{\defenseCouncilNumber}     % Защита, номер диссертационного совета
{Д\,212.125.12}
\newcommand{\defenseCouncilTitle}      % Защита, учреждение диссертационного совета
{ФГБОУ ВО <<Московский авиационный институт (национальном исследовательском университете)>>}
\newcommand{\defenseCouncilAddress}    % Защита, адрес учреждение диссертационного совета
{125993, Москва, A-80, ГСП-3, Волоколамское шоссе, д. 4}
\newcommand{\defenseCouncilPhone}      % Телефон для справок
{+7~499~158-43-55}

\newcommand{\defenseSecretaryFio}      % Секретарь диссертационного совета, ФИО
{Старков А.\,В.}
\newcommand{\defenseSecretaryRegalia}  % Секретарь диссертационного совета, регалии
{кандидат технических наук, доцент}    % Для сокращений есть ГОСТы, например: ГОСТ Р 7.0.12-2011 + http://base.garant.ru/179724/#block_30000

\newcommand{\synopsisLibrary}          % Автореферат, название библиотеки
{ФГБОУ ВО <<Московский авиационный институт (национальный исследовательский университет)>>
по адресу: 125993, Москва, A-80, ГСП-3, Волоколамское шоссе, д. 4,
а также на сайте института по адресу https://mai.ru/events/defence/index.php?ELEMENT\_ID=110619}
\newcommand{\synopsisDate}             % Автореферат, дата рассылки
{<<\uline{\hspace{1.2em}}>> \uline{\hspace{6em}} 2020 года}

% To avoid conflict with beamer class use \providecommand
\providecommand{\keywords}%            % Ключевые слова для метаданных PDF диссертации и автореферата
{}

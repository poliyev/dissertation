\textbf{Актуальность темы.}
На сегодняшний день взаимодействие человека с компьютерными системами через управление речевыми командами является одним из самых удобных и перспективных форматов.

% версия для автореферата
%Первая попытка конструирования системы автоматического распознавания речи была сделана в 1952 году, а уже в 1970-е годы исследования в области распознавания речи достигли значительных успехов.
%До настоящего времени распознавание речи совершенствовалось, а словарь распознаваемых слов вырос до нескольких десятков тысяч.
%Применение быстрых методов декодирования позволило производить распознавание в реальном времени.

% версия для диссертации
Первая попытка конструирования системы автоматического распознавания речи была сделана в 1952 году в Bell Laboratories, США~\cite{davis1952automatic}.
Система с хорошим уровнем точности распознавала цифры от нуля до девяти, произнесённые диктором через телефонный автомат.
Значительные улучшения качества в области распознавания речи были достигнуты в 70-ых годах.
В то время технологии автоматического распознавания отдельных команд основывалась на работах Itakura в США~\cite{itakura1975minimum}, Sakoe и Chiba в Японии~\cite{sakoe1978dynamic} и Величкина и Загоруйка в СССР~\cite{velichko1970automatic}.
Советские учёные производили улучшения методов распознавания с помощью эталона.
Применение подхода динамического программирования было отличительной особенностью японского исследования.
Работа Itakura раскрыла метод кодирования линейного предсказания (Linear Predictive Coding, LPC), который успешно использовался в распознавании сигналов с низким битрейтом (количество битов информации, передаваемых в секунду).
В AT\&T Bell Laboratories были построены распознающие системы, обработка акустического сигнала в которых была основана на LPC анализе, а процесс распознавания проходил с использованием метода динамической трансформации времени (Dynamic Time Warping, DTW).
В 1980-х годах от подходов, основанных на применении эталонов, научные работы в области распознавания речи перешли к моделированию статистическими методами.
Использовались скрытые модели Маркова (Hidden Markov Models, HMM).
Работы Бейкера~\cite{baker1990stochastic} были одними из первых, в которых для решения задачи распознавания речи были применены HMM.
С 1990-х годов распознавание речи несколько усовершенствовалось.
Словарь распознаваемых слов вырос до нескольких десятков тысяч.
Использование быстрых методов декодирования позволило производить распознавание в реальном времени.
В современных дикторозависимых системах, распознающих отдельные слова, количество которых достигает двадцати тысяч слов, ошибки составляют менее 0.1~\%~\cite{das1993influence}.
И около 5~\% ошибок в независимых от диктора системах, которые распознают слитную речь из тысячи слов~\cite{aubert1993continuous}.

В современных системах применяются 3 основные группы методов распознавания речи.
Первая группа --- это скрытые марковские модели.
В них входная речь рассматривается как последовательность фонем с определёнными вероятностями перехода.
Распознавание производится через поиск наиболее вероятной последовательности фонем для данного входного сигнала.
Вторая --- это методы, основанные на сравнении с эталоном.
Для каждого слова из словаря некоторым образом составляется эталон.
При распознавании выбирается то слово, эталон которого наиболее близок к входному сигналу.
Третья группа методов основана на искусственных нейронных сетях.
Суть методов состоит в нахождении такой решающей функции, которая по входному сигналу может определить его принадлежность к определённому классу.
Искусственные нейронные сети построены по принципу организации биологических нейронных сетей и хорошо справляются с широким спектром задач.

В данной работе решается задача повышения вероятности правильных распознаваний и снижения влияния акустических шумов путём разработки и совершенствования алгоритмов распознавания команд речевого интерфейса пилота для управления бортовым оборудованием современных самолётов.
По сравнению с обычной задачей распознания речи к речевому интерфейсу кабины пилота предъявляются следующие требования:

\begin{itemize}
	\item распознавание ограниченного словаря из слов или фраз;
	\item компактность, автономность, высокое быстродействие;
	\item хорошее качество распознавания в условиях сильного шума.
\end{itemize}

С учётом этих требований широко используемые скрытые марковские модели не подходят из-за низкого качества распознавания в условиях шума \cite{korsun2013experimental}.
Остальные две группы методов в настоящий момент не обеспечивают необходимой надёжности распознавания.
По этой причине тема настоящей работы, направленной на совершенствование методов распознавания речевых команд с помощью сравнения с эталоном и с использованием нейронных сетей, является актуальной.
Исследования, выполненные в рамках данной работы, направлены на решение таких практически значимых и актуальных задач, как предобработка входящего сигнала путём выделения однородных частей, улучшение качества эталонов с помощью выделения в них главных компонент и использование систем распознавания из нескольких эталонов.
В работе также проведено обширное экспериментальное исследование всех разработанных методов на различных наборах входных данных с несколькими уровнями шума.

\textbf{Объект и предмет исследования.}
В работе в качестве объекта исследования рассматриваются речевые команды, а предметами исследования являются методы и алгоритмы распознавания речевых команд.

\textbf{Целью} работы является повышение вероятности правильных распознаваний и снижение влияния акустических шумов, путём разработки алгоритмического обеспечения для распознавания команд речевого интерфейса кабины пилота в виде отдельных слов и фраз.
За рамками работы остались выбор оптимального состава команд и их интерпретация.

Для достижения поставленной цели решаются следующие научно-технические задачи:
\begin{itemize}
	\item анализ статистических свойств речевых команд и их нормализация;
	\item разработка алгоритмов предварительного разбиения записей на однородные части;
	\item разработка алгоритмов исключения шума и выделения наиболее значимых компонент в эталоне;
	\item исследование статистических закономерностей верного и неверного распознавания речевых команд и их использование для уменьшения количества ошибок;
	\item разработка алгоритмов использования нескольких эталонов одного слова для улучшения качества распознавания;
	\item исследование современных типов и архитектур искусственных нейронных сетей глубокого обучения для применения в задаче распознавания речевых команд.
\end{itemize}

\textbf{Методология и методы исследования.}
Основными методами исследования, используемыми в работе, являются: анализ данных, цифровая обработка сигналов, теория вероятностей, математическая статистика, численная оптимизация, проектирование программных средств.

\textbf{Научная новизна} заключается в разработке совокупности алгоритмов, обеспечивающих повышение вероятности правильных распознаваний команд речевого интерфейса кабины пилота:
\begin{itemize}
	\item алгоритм разбиения речевых команд на фонетически однородные части на основе модифицированного метода динамического программирования;
	\item алгоритм оптимизации эталонов на основе метода главных компонент;
	\item алгоритм оптимизации размерности параметрических портретов с использованием полиномов Чебышёва;
	\item алгоритм распознавания команд по нескольким эталонам с использование байесовского подхода и метода комитетов;
	\item алгоритм распознавания команд свёрточными нейронными сетями, способных обучаться на выборках малого размера.
\end{itemize}

\textbf{Практическая значимость.}
Полученная в результате работы совокупность алгоритмов обеспечивает высокую точность распознавания речевых команд при различных уровнях шума, в том числе с учётом случая статически неустойчивого самолёта.
Результаты работы могут быть применены в учебном процессе и в ходе разработки алгоритмического обеспечения речевого интерфейса пилота для таких задач, как отображение информации, выбор частоты радиооборудования, прокладка маршрута, управление системой опознавания и датчиками, запрос запаса топлива.

\textbf{Положения, выносимые на защиту:}
\begin{enumerate}[label={\arabic*)}]
	\item Разработан алгоритм разбиения речевых команд на фонетически однородные части, отличающийся от существующих применением модифицированного метода динамического программирования.
	\item Разработан алгоритм оптимизации эталонов, отличающийся от существующих тем, что искомый эталон формируется как линейная комбинация главных компонент, оптимизирующая заданный критерий качества.
	\item Разработан алгоритм оптимизации размерности параметрических портретов, отличающийся выделением наиболее значимых составляющих с использованием полиномов Чебышёва.
	\item Разработан алгоритм распознавания команд по нескольким эталонам, отличающийся применением последовательного оценивания с расчётом апостериорных байесовских вероятностей.
	\item Разработан алгоритм распознавания команд свёрточными нейронными сетями глубокого обучения, отличающийся от существующих обучением на выборке малого размера.
\end{enumerate}

\textbf{Достоверность результатов} обеспечивается корректным применением математической статистики, методов идентификации и анализа данных, подтверждением полученных теоретических результатов с помощью экспериментов, а также сравнением с известными результатами, ранее полученными другими авторами.

% версия для автореферата
%\textbf{Апробация работы.}
%Основные результаты исследования докладывались на следующих конференциях:
%\begin{enumerate}[label={\arabic*)}]
%	\item Доклад <<Получение оптимального эталона с помощью метода главных компонент>> на Всероссийской научно-технической конференции <<XII Научные чтения по авиации, посвящённые памяти Н.Е. Жуковского>> (Москва, 2015) \cite{poliyev2015pca}.
%	\item Доклад <<Алгоритм разбиения слов на однородные части в интересах разработки речевого интерфейса бортового оборудования>> на Восьмом Международном Аэрокосмическом Конгрессе IAC'15 (Москва, 2015) \cite{poliyev2015split}.
%	\item Доклад <<Разработка модифицированного алгоритма динамического программирования для разбиения слов на однородные части>> на Всероссийской научно-технической конференции <<XIII Научные чтения по авиации, посвящённые памяти Н.Е. Жуковского>> (Москва, 2016) \cite{poliyev2016dynamic}.
%	\item Доклад <<Определение оптимального разбиения слова на однородные участки на основе матрицы корреляционного портрета>> на Юбилейной Всероссийской научно-технической конференции <<Авиационные системы в XXI веке>> (Москва, 2016) \cite{poliyev2016split, poliyev2017split}.
%	\item Доклад <<Разработка метода анализа фонетически однородных частей слов естественного языка>> на Второй Международной научно-практической конференции <<Эрго-2016: Человеческий фактор в сложных технических системах и средах>> (Санкт-Петербург, 2016) \cite{poliyev2016natural}.
%	\item Доклад <<The algorithm of an optimal word pattern synthesis using principal component analysis>> на международном семинаре Workshop on Contemporary Materials and Technologies in the Aviation Industry --- CMTAI (Москва, 2016) \cite{poliyev2016pca}.
%	\item Доклад <<Применение формулы Байеса для распознавания слов с использованием нескольких эталонов>> на Всероссийской научно-технической конференции <<Навигация, наведение и управление летательными аппаратами>> (Москва, 2017) \cite{poliyev2017bayes}.
%	\item Доклад <<Разработка алгоритма распознавания слов в условиях шума на основе свёрточных нейронных сетей>> на Девятом Международном Аэрокосмическом Конгрессе IAC'18 (Москва, 2018) \cite{poliyev2018cnn}.
%	\item Доклад <<Распознавание речевых команд на основе свёрточных нейронных сетей>> на Всероссийской научно-технической конференции <<Моделирование авиационных систем>> (Москва, 2018) \cite{poliyev2018cnn2}.
%\end{enumerate}
%
%\textbf{Публикации.}
%По теме диссертации автором опубликовано 4 научных работы \cite{korsun2016automatic,poliyev2017pca,korsun2018usage,korsun2018optimal}: 3 из них в изданиях из списка, рекомендованного ВАК РФ \cite{korsun2016automatic,poliyev2017pca,korsun2018usage}, и 2 из них в изданиях, входящих в базу Scopus и базу Web of Science \cite{korsun2016automatic,korsun2018optimal}.

% версия для диссертации
\textbf{Апробация работы.}
Основные результаты исследования докладывались на следующих конференциях:
\begin{enumerate}[label={\arabic*)}]
	\item Доклад на Всероссийской научно-технической конференции <<XII Научные чтения по авиации посвящённые памяти Н.Е. Жуковского>> (Москва, 17 апреля 2015 года).
	Тема доклада: <<Получение оптимального эталона с помощью метода главных компонент>>.
	Текст доклада напечатан в сборнике докладов конференции \cite{poliyev2015pca}.
	\item Доклад на Восьмом Международном Аэрокосмическом Конгрессе IAC'15 (Москва, 28--31 августа 2015 года).
	Тема доклада: <<Алгоритм разбиения слов на однородные части в интересах разработки речевого интерфейса бортового оборудования>>.
	Текст доклада напечатан в сборнике докладов конференции \cite{poliyev2015split}.
	\item Доклад на Всероссийской научно-технической конференции <<XIII Научные чтения по авиации посвящённые памяти Н.Е. Жуковского>> (Москва, 14 апреля 2016 года).
	Тема доклада: <<Разработка модифицированного алгоритма динамического программирования для разбиения слов на однородные части>>.
	Текст доклада напечатан в сборнике докладов конференции \cite{poliyev2016dynamic}.
	\item Доклад на Юбилейной Всероссийской научно-технической конференции <<Авиационные системы в XXI веке>> (Москва, 26 мая 2016 года).
	Тема доклада: <<Определение оптимального разбиения слова на однородные участки на основе матрицы корреляционного портрета>>.
	Текст доклада напечатан в сборнике докладов конференции \cite{poliyev2016split, poliyev2017split}.
	\item Доклад на Второй Международной научно-практической конференции <<Эрго-2016: Человеческий фактор в сложных технических системах и средах>> (Санкт-Петербург, 6--9 июля 2016 года).
	Тема доклада: <<Разработка метода анализа фонетически однородных частей слов естественного языка>>.
	Текст доклада напечатан в сборнике докладов конференции \cite{poliyev2016natural}.
	\item Доклад на международном семинаре Workshop on Contemporary Materials and Technologies in the Aviation Industry --- CMTAI (Москва, 15--16 декабря 2016 года).
	Тема доклада: <<The algorithm of an optimal word pattern synthesis using principal component analysis>>.
	Текст доклада напечатан в сборнике докладов конференции \cite{poliyev2016pca}.
	\item Доклад на Всероссийской научно-технической конференции <<Навигация, наведение и управление летательными аппаратами>> (Москва, 21--22 сентября 2017 года).
	Тема доклада: <<Применение формулы Байеса для распознавания слов с использованием нескольких эталонов>>.
	Текст доклада напечатан в сборнике докладов конференции \cite{poliyev2017bayes}.
	\item Доклад на Девятом Международном Аэрокосмическом Конгрессе IAC'18 (Москва, 28--31 августа 2018 года).
	Тема доклада: <<Разработка алгоритма распознавания слов в условиях шума на основе свёрточных нейронных сетей>>.
	Текст доклада напечатан в сборнике докладов конференции \cite{poliyev2018cnn}.
	\item Доклад на Всероссийской научно-технической конференции <<Моделирование авиационных систем>> (Москва, 21--22 ноября 2018 года).
	Тема доклада: <<Распознавание речевых команд на основе свёрточных нейронных сетей>>.
	Текст доклада напечатан в сборнике докладов конференции \cite{poliyev2018cnn2}.
\end{enumerate}

\textbf{Публикации.}
По теме диссертации автором опубликовано 4 научных работы \cite{korsun2016automatic,poliyev2017pca,korsun2018usage,korsun2018optimal}: 3 из них в изданиях из списка, рекомендованного ВАК РФ \cite{korsun2016automatic,poliyev2017pca,korsun2018usage}, и 2 из них в изданиях, входящих в базу Scopus и базу Web of Science \cite{korsun2016automatic,korsun2018optimal}.
\begin{enumerate}[label={\arabic*)}]
	\item Статья <<Автоматическое выделение фонетически однородных участков в словах естественного языка на основе многопараметрической оптимизации>> в журнале <<Известия Российской академии наук. Теория и системы управления>>, 2016 год, № 4, страницы 145–-154 \cite{korsun2016automatic}.
	\item Статья <<Разработка алгоритма синтеза оптимальных эталонов на основе метода главных компонент>> в журнале <<Cloud of science>>, 2017 год, № 4, страницы 650--661 \cite{poliyev2017pca}.
	\item Статья <<Использование нескольких эталонов при распознавании речи: формула Байеса и метод комитетов>> в журнале <<Вестник компьютерных и информационных технологий>>, 2018 год, № 1, страницы 14--23 \cite{korsun2018usage}.
	\item Статья <<Optimal pattern synthesis for speech recognition based on principal component analysis>> в журнале <<IOP Conference Series: Materials Science and Engineering>>, 2018 год, № 312, страницы 12--14 \cite{korsun2018optimal}.
\end{enumerate}

